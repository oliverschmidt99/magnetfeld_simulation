% --- Beginn deines Themas ---
\section{Fremdfeldbeeinflussung bei Messstromwandlern}

\begin{frame}
    \centering
    \vfill
    {\Huge Fremdfeldbeeinflussung bei Messstromwandlern}
    \vfill
\end{frame}


\begin{frame}
    \frametitle{Gliederung}
    \begin{itemize}
        \Large
        \item Messstromwandler im Überblick – und wo das Problem liegt
        \item Wie wurde das untersucht? (Simulation mit FEMM)
        \item Was zeigen die Messungen?
    \end{itemize}
\end{frame}



% -----------------------------------------------------------------
% 1. Problemstellung und Grundlagen
% -----------------------------------------------------------------
\section*{Grundlagen und Herausforderung}

\begin{frame}
    \frametitle{Wie funktionieren Messstromwandler eigentlich?}
    \begin{beamercolorbox}[wd=\textwidth, sep=1ex, rounded=true, shadow=true]{examplebox}
        \begin{columns}[T]
            \begin{column}{0.3\textwidth}
                \textbf{Funktionsweise}
                \begin{itemize}
                    \item Ein Wandler funktioniert nach dem Transformatorprinzip.
                    \item Galvanische Trennung, da man den Lastkreis von dem Messkreis trennt.
                \end{itemize}
            \end{column}

            \begin{column}{0.55\textwidth}
                \includegraphics[width=\linewidth]{03_Ressourcen//erwärmungsprüfstand/erwärmungsprüfstand_wandler_2.pdf}
            \end{column}
        \end{columns}
    \end{beamercolorbox}

\end{frame}


\begin{frame}
    \frametitle{Einsatzorte in der Niederspannungsschaltanlage}
    \begin{beamercolorbox}[wd=\textwidth, sep=1.2ex, rounded=true, shadow=true]{examplebox}
        \begin{itemize}
            \item Direkte Montage auf Eingangs- oder Abgangsschienen
            \item Überwachung und Analyse der Betriebsströme
            \item Erfassung von Messwerten für das Verrechnungswesen
        \end{itemize}
    \end{beamercolorbox}
\end{frame}




\begin{frame}
    \frametitle{Warum entsteht eine Fremdfeldbeeinflussung?}
    \begin{beamercolorbox}[wd=\textwidth, sep=1ex, rounded=true, shadow=true]{examplebox}
        \begin{columns}[T]
            \begin{column}{0.55\textwidth}
                \textbf{Schaltanlage mit 6300 A}
                \begin{figure}
                    \centering
                    \includegraphics[width=1\linewidth]{03_Ressourcen//erwärmungsprüfstand/6300A_Wandler_1.jpg}
                \end{figure}

            \end{column}

            \begin{column}{0.3\textwidth}
                \begin{figure}
                    \centering
                    \includegraphics[width=0.9\linewidth]{03_Ressourcen//erwärmungsprüfstand/6300A_Wandler_2.jpg}
                \end{figure}
            \end{column}
        \end{columns}
    \end{beamercolorbox}
\end{frame}



\begin{frame}
    \frametitle{Warum entsteht eine Fremdfeldbeeinflussung?}
    \begin{beamercolorbox}[wd=0.89\textwidth, sep=1ex, rounded=true, shadow=true]{examplebox}
        \textbf{Schaltanlage mit 1250 A}
        \begin{figure}
            \centering
            \includegraphics[width=0.89\linewidth]{2000A_Wandler.jpg}
        \end{figure}
    \end{beamercolorbox}
\end{frame}


\begin{frame}
    \frametitle{Warum entsteht eine Fremdfeldbeeinflussung?}
    \begin{beamercolorbox}[wd=1\textwidth, sep=1ex, rounded=true, shadow=true]{examplebox}
        \begin{itemize}
            \item Jeder Leiter erzeugt ein Magnetfeld.
            \item Wandler soll \textbf{nur} das Feld des umschlossenen Leiters messen.
            \item Problem in kompakten Anlagen: Schienen sind \textbf{dicht beieinander}.
            \item Magnetfeld des Nachbarleiters wirkt als \textbf{Fremdfeld} und durchdringt den Kern des Wandlers.
            \item Dies überlagert sich mit dem Messfeld.
            \item \textbf{Folge:} Messfehler/Verfälschtes Messergebnis.
        \end{itemize}
    \end{beamercolorbox}
\end{frame}







% -----------------------------------------------------------------
% 2. Simulation (FEMM)
% -----------------------------------------------------------------
\section*{Simulation mit FEMM}

\begin{frame}
    \frametitle{Simulationswerkzeug FEMM}
    \begin{beamercolorbox}[wd=\textwidth, sep=1ex, rounded=true, shadow=true]{examplebox}
        \begin{columns}[c]
            \begin{column}{0.45\textwidth}
                \begin{itemize}
                    \item \textbf{FEMM} = \textbf{F}inite \textbf{E}lement \textbf{M}ethod \textbf{M}agnetics.
                    \item Es ist ein kostenloses Open-Source-Softwarepaket.
                    \item Dient zur Lösung von 2D-Problemen mittels der Finiten-Elemente-Methode (FEM).
                \end{itemize}
            \end{column}
            \begin{column}{0.5\textwidth}
                \begin{figure}
                    \centering
                    \includegraphics[width=1\linewidth]{03_Ressourcen//erwärmungsprüfstand/grafik.png}
                \end{figure}
            \end{column}
        \end{columns}
        \centering
        \textbf{Quelle: www.femm.info}
    \end{beamercolorbox}
\end{frame}

\begin{frame}{Schaubild aus FEMM}
    \begin{figure}
        \centering
        \includegraphics[width=1\linewidth]{03_Ressourcen//erwärmungsprüfstand/femm-sim.png}
    \end{figure}
\end{frame}






\begin{frame}
    \begin{beamercolorbox}[wd=\textwidth, sep=1ex, rounded=true, shadow=true]{examplebox}
        \frametitle{Aufbau des Modells und Optimierungsansätze}
        \textbf{Aufbau des Simulationsmodells (2D-Wandler):}
        \begin{itemize}
            \item \textbf{Herausforderung:} FEMM ist für \textbf{stationäre} Fälle ausgelegt.
            \item \textbf{Anforderung:} Simulation eines transienten \textbf{AC-Phasenablaufs} (z.B. 50 Hz Sinus).
            \item \textbf{Lösung:} \textbf{Quasi-stationäre Simulation} (Simulation von diskreten Zeitpunkten).

            \item \textbf{1. Modellierung:} Geometrie des Wandlers definieren.

            \item \textbf{2. Äußere Schleife:} Iteration über Hauptparameter, z. B.:
                  \begin{itemize}
                      \item Unterschiedliche Stromamplituden ($\hat{I}$) oder Lastzustände.
                  \end{itemize}

            \item \textbf{3. Innere Schleife (Phasendurchlauf):} Für jeden Parameter aus (2):
                  \begin{itemize}
                      \item Iteration über diskrete Zeitpunkte/Winkel einer Periode (z.B. $0^\circ \dots 360^\circ$).
                      \item Pro Schritt: Berechnung der \textit{momentanen} Phasenströme (z.B. $I_A, I_B, I_C$)
                      \item Pro Schritt: Ausführung der \textit{stationären} FEMM-Simulation.
                      \item Speicherung der Ergebnisse (z. B. Induktivität, Verluste).
                  \end{itemize}
        \end{itemize}
    \end{beamercolorbox}
\end{frame}


\begin{frame}
    \begin{beamercolorbox}[wd=\textwidth, sep=1ex, rounded=true, shadow=true]{examplebox}
        \begin{figure}
            \centering
            \includegraphics[width=0.75\textwidth]{03_Ressourcen//erwärmungsprüfstand/femm_website_config.png}
        \end{figure}
    \end{beamercolorbox}
\end{frame}
% -----------------------------------------------------------------
% 3. Experimentelle Validierung (Messungen)
% -----------------------------------------------------------------
\section*{Was zeigen die Messungen?}

\begin{frame}
    \frametitle{Aufbau des Versuchs und Vorbereitung}
    \begin{itemize}
        \item Nutzung des \textbf{Erwärmungsprüfstands} als Hochstromquelle (wie auf Folie 26 gezeigt), um reale Betriebsströme (z.B. 2000 A) zu erzeugen.
        \item \textbf{Versuchsaufbau:}
              \begin{itemize}
                  \item Installation eines Sammelschienensystems (z.B. L1 und L2).
                  \item Montage des Messstromwandlers (Prüfling) auf Schiene L1.
                  \item Schiene L2 dient als Störquelle (Fremdleiter).
                  \item Präzise Messung des Sekundärstroms des Wandlers mittels eines Referenzmessgeräts (z.B. Kocos Artest 600 oder ein Präzisions-Leistungsmesser).
              \end{itemize}
        \item \textbf{Vorbereitung / Durchführung:}
              \begin{itemize}
                  \item \textbf{Referenzmessung:} Strom nur auf L1 (Messstrom), L2 ist AUS. $\rightarrow$ Bestimmung des "wahren" Werts.
                  \item \textbf{Störmessung:} Strom auf L1 (Messstrom) UND auf L2 (Fremdstrom). $\rightarrow$ Bestimmung des "verfälschten" Werts.
                  \item Berechnung des Messfehlers.
                  \item Systematische Variation des Abstands und Test der Abschirmbleche (ST37, Mu-Metall) analog zur Simulation.
              \end{itemize}
    \end{itemize}
    \begin{center}
        \textit{(Hier war das Foto des realen Versuchsaufbaus am Prüfstand zu sehen.)}
    \end{center}
\end{frame}

\begin{frame}
    \frametitle{Ergebnisse und Erkenntnisse}
    Die experimentellen Messungen bestätigen die FEMM-Simulationen sehr gut:
    \begin{itemize}
        \item \textbf{Abhängigkeit vom Abstand:}
              \begin{itemize}
                  \item Der Messfehler durch das Fremdfeld ist \textbf{extrem stark abstandsabhängig}.
                  \item Bei geringen Abständen (wie sie in kompakten Anlagen üblich sind) steigt der Fehler überproportional an und kann die zulässigen Grenzwerte der Genauigkeitsklasse (z.B. Klasse 1 $\rightarrow \pm 1\%$) deutlich überschreiten.
              \end{itemize}
        \item \textbf{Wirksamkeit der Abschirmung:}
              \begin{itemize}
                  \item \textbf{Ohne Schirmung:} Hohes Risiko für signifikante Messfehler.
                  \item \textbf{Mit Schirmung (ST37):} Ein einfaches Baustahlblech bietet bereits eine gute Schutzwirkung und reduziert den Fehler erheblich (z.B. um 70\%).
                  \item \textbf{Mit Schirmung (Mu-Metall):} Hochpermeable Materialien (teuer!) eliminieren den Fremdfeldeinfluss fast vollständig.
              \end{itemize}
        \item \textbf{Zentrale Erkenntnis:}
              \begin{itemize}
                  \item Die Positionierung von Messstromwandlern ist \textbf{nicht trivial}.
                  \item Für genaue Messungen (besonders wichtig für Energiemanagement nach ISO 50001) müssen Mindestabstände eingehalten oder \textbf{konstruktive Abschirmmaßnahmen} vorgesehen werden.
              \end{itemize}
    \end{itemize}
    \begin{center}
        \textit{(Hier waren die beiden Ergebnis-Diagramme: 1. Messfehler vs. Abstand, 2. Messfehler vs. Art der Abschirmung.)}
    \end{center}
\end{frame}